% SPDX-FileCopyrightText: 2025 IObundle
%
% SPDX-License-Identifier: MIT

\documentclass{ug}

%
% Packages and configuration
%

\usepackage{xltabular}

\usepackage{hyperref}
\hypersetup{bookmarksnumbered=true}
% Set up hyperlink colors
\hypersetup{
    colorlinks=true, % false: boxed links; true: colored links
    linkcolor=blue, % color of internal links
    citecolor=blue, % color of citations
    urlcolor=blue   % color of external links
}


\usepackage{listings}
\usepackage{xcolor}
% Define colors
\definecolor{keywordcolor}{RGB}{0, 0, 255} % Blue for keywords
\definecolor{commentcolor}{RGB}{0, 128, 0} % Green for comments
\definecolor{stringcolor}{RGB}{255, 0, 0} % Red for strings
\definecolor{backgroundcolor}{RGB}{245, 245, 245} % Light gray background

% Set up listings
\lstdefinestyle{Python}{
    backgroundcolor=\color{backgroundcolor},
    basicstyle=\ttfamily,
    keywordstyle=\color{keywordcolor},
    commentstyle=\color{commentcolor},
    stringstyle=\color{stringcolor},
    numbers=left,
    numberstyle=\tiny,
    stepnumber=1,
    numbersep=5pt,
    showstringspaces=false,
    tabsize=4,
    breaklines=true
}
% Set the default style for all lstlisting environments
\lstset{style=Python}

% Support unicode chars for drawing directory trees (├ ─ └)
\usepackage{pmboxdraw}

%
% Document content
%

% SPDX-FileCopyrightText: 2025 IObundle
%
% SPDX-License-Identifier: MIT

\title{%
\Huge {\bf \input{name}} \\
 \vspace*{3cm}
\Large {\bf \input{description}} \\
}

\header{\input{name}, \input{description}}

\date{\today}
\category{User Guide, \input{\NAME_version.tex}, Build \input{shortHash.tex}}

\input{color}

\begin{document}

\maketitle
\pagenumbering{gobble}

\vspace*{\fill}
User Guide, \input{\NAME_version.tex}, Build \input{shortHash.tex}
\hspace*{\fill} \includegraphics[keepaspectratio,scale=.7]{Logo.png}

\cleardoublepage
\pagenumbering{roman}
\setcounter{page}{1}
\input{revhist}
\cleardoublepage
\tableofcontents
\clearpage
\listoftables
\clearpage
\listoffigures
\cleardoublepage
\pagenumbering{arabic}
\setcounter{page}{1}

%
% Introduction
%
\section{Introduction}
\label{sec:intro}
\input{intro}
%\input{symb}

\subsection{What Is Py2HWSW?}
\label{sec:whatispy2}
\input{whatispy2}

\subsection{What Is Py2HWSW For?}
\label{sec:purpose}
\input{purpose}

\subsection{What Problem Does Py2HWSW Solve?}
\label{sec:problem}
\input{problem}

\subsection{What Design Principles Underlie Py2HWSW?}
\label{sec:principles}
\input{principles}

\subsection{How Does Py2HWSW Accomplish Its Goals?}
\label{sec:how}
\input{how}

%
% Getting Started
%
\ifdefined\SECTIONCLEARPAGE
\clearpage
\fi
\section{Getting Started}
\label{sec:gs}

\subsection{Setup Directory}
\label{sec:setup_dir}
\input{setup_dir}

\subsection{Create An AND Gate Core: iob\_and}
\label{sec:iob_and}
\input{iob_and}

\subsection{Setup And Build}
\label{sec:setup_build}
\input{setup_build}

\subsection{Installation}
\label{sec:installation}
\input{installation}

\subsection{Basic Usage}
\label{sec:basic_usage}
\input{basic_usage}

\subsection{Universal Testbench}
\label{sec:utb}
\input{utb}

%
% How It Works
%
\ifdefined\SECTIONCLEARPAGE
\clearpage
\fi
\section{How It Works}
\label{sec:how_it_works}

This section gives a detailed description of the Py2HWSW framework.

\subsection{Overview}
\label{sec:py2_overview}
\input{overview}

\subsection{Technical Details}
\label{sec:py2_technical_details}
\input{technical_details}

\subsection{Setup Flow Chart}
\label{sec:py2_flow_chart}

Figure~\ref{fig:py2_flow_chart} presents a high-level flow chart of the Py2HWSW setup procedure.

\begin{figure}[H]
  \centering {\includegraphics[width=\columnwidth]{py2_flow_chart.pdf}}
  \vspace{-0.7cm}
  \caption{High-Level Flow Chart of Py2HWSW Setup Procedure}
  \label{fig:py2_flow_chart}
\end{figure}

\subsection{Standard Interfaces}
\label{sec:py2_standard_interfaces}
% SPDX-FileCopyrightText: 2025 IObundle
%
% SPDX-License-Identifier: MIT

The Py2HWSW framework provides the following two standard interfaces:
\begin{enumerate}
  \item \textbf{Python Parameters}: Core "setup" function receives information from Py2HWSW via a dictionary in its first argument, referred to as \textit{Python Parameters}.
  \item \textbf{Core Dictionary}: Core "setup" function returns a core description dictionary to Py2HWSW, referred to as \textit{Core Dictionary}.
\end{enumerate}

The core's "setup" function is the python function defined by the user in the <core\_name>.py file.

If the core is described by a JSON file, then the \textit{Python Parameters} interface is not available.
The JSON file gives a dictionary to Py2HWSW, similar to the python dictionary of the "setup" function.
This allows the user to use external tools to generate cores in JSON format.

%
% Python parameters
%

The \textit{Python Parameters} received by the core's "setup" function is a dictionary containing both parameters passed by its issuer and standard parameters passed by Py2HWSW.
Each key, value pair in the dictionary is a \textit{Python Parameter}.
The value of the python parameter may be of any data type.

\begin{xltabular}{\textwidth}{|l|l|X|}

  \hline
  \rowcolor{iob-green}
  {\bf Name} & {\bf Data Type} & {\bf Description}  \\ \hline \hline

  \input py2hwsw_py_params_tab

  \caption{Standard \textit{Python Parameters} passed by Py2HWSW to every core's "setup" function.}
\end{xltabular}
\label{py2hwsw_py_params_tab}

The standard python parameters passed by Py2HWSW are listed in Table~\ref{py2hwsw_py_params_tab}.

The python parameters supported by each core is available in the respective core's user guide, as long as they have the \textit{Python Parameters} attribute defined.
Instructions on how to build a core's user guide can be found in Section~\ref{sec:core_lib}. 


%
% Core dictionary
%

\begin{xltabular}{\textwidth}{|l|l|X|}
  \hline
  \rowcolor{iob-green}
  % TODO: The "Data Type" column should specify what the user should input, instead of the internal object used by py2hwsw.
  {\bf Name} & {\bf Data Type} & {\bf Description}  \\ \hline \hline
  \endhead

  \input py2hwsw_attributes_tab

  \caption{Table of supported Py2HWSW attributes in the \textbf{Core Dictionary}. The \textit{Data Type} column specifies the type of internal object that the Py2HWSW will convert the attribute's value to (usually the user inputs a string, list, or dictionary value and then py2 converts it to an internal object).}
  \label{py2hwsw_attributes_tab}
\end{xltabular}

The list of attributes supported by the Py2HWSW framework is given in Table~\ref{py2hwsw_attributes_tab}.
If a core provides a dictionary with keys not listed in Table~\ref{py2hwsw_attributes_tab}, then the Py2HWSW framework will raise an error.
Each key, value pair in the dictionary is a \textit{Core Attribute}.
The data type of the core attribute may be of any data type, but are usually a string, list, or dictionary.
If the data type is a string, it may also represent an object using Py2HWSW's \textit{Short Notation}.
%~\ref{sec:short_notation}


\subsection{Block hierarchy}
\label{sec:py2_block_hierarchy}

Figure~\ref{fig:py2_superblocks_subblocks} presents an example block hierarchy for a Py2HWSW project.
Superblocks are only used if they are superblocks of the project's top module or of one of its wrappers.

\begin{figure}[H]
  \centering {\includegraphics[width=\columnwidth]{superblocks_subblocks.pdf}}
  \vspace{-0.7cm}
  \caption{Block Hierarchy of a Py2HWSW Project}
  \label{fig:py2_superblocks_subblocks}
\end{figure}

\subsection{Main launch script: py2hwsw.py}
\label{sec:launch_script}
\input{launch_script}

\subsection{Simulate with Verilator}
\label{sec:verilator}
\input{verilator}

\subsection{Deliver an IP core}
\label{sec:deliver}
\input{deliver.tex}
%
% Py2HWSW classes
%
\section{Py2HWSW Classes}
\label{sec:py_classes}

\subsection{Main class for core representation: iob\_core.py}
\label{sec:iob_core}
\input{iob_core}

\subsection{Configuration class: iob\_conf.py}
\label{sec:iob_conf}
\input{iob_conf}

\subsection{Signal class: iob\_signal.py}
\label{sec:iob_signal}
\input{iob_signal}

\subsection{Wire class: iob\_wire.py}
\label{sec:iob_wire}
\input{iob_wire}

\subsection{Port class: iob\_port.py}
\label{sec:iob_port}
\input{iob_port}

\subsection{Interface class: if\_gen.py}
\label{sec:if_gen}
\input{if_gen}

\subsection{Special cases}
\label{sec:special_cases}
% SPDX-FileCopyrightText: 2025 IObundle
%
% SPDX-License-Identifier: MIT

Most of the cores provived by the py2hwsw's library are built using the standard interfaces mentioned in section~\ref{sec:py2_standard_interfaces}.

However, there are some cores that due to limitations of the standard interfaces, rely instead on internal py2hwsw methods for extra features.
The following list describes the cores don't rely solely on the standard interfaces.

\begin{itemize}
\item \textbf{iob\_system}: This core uses the `is\_system` attribute to enable an internal py2hwsw method that automatically fixes the address widths of the cbus interfaces of the system's peripherals.
\item \textbf{iob\_csrs}: The py2hwsw tool contains an internal method to automatically search for the "iob\_csrs" subblock and insert a "\textless prefix\textgreater \_cbus\_s" port on the issuer core of this subblock. It then connects this newly created "\textless prefix\textgreater \_cbus\_s" port of the issuer core to the iob\_csrs "control\_if\_s" port. The '\textless prefix\textgreater ' is replaced by instance name of iob\_csrs subblock.
\end{itemize}




\subsection{Core library}
\label{sec:core_lib}
\input{core_lib}

%
% How To Use
%
\ifdefined\SECTIONCLEARPAGE
\clearpage
\fi
\section{How To Use}
\label{sec:usage}

\subsection{Setup}
\label{sec:setup}
\input{setup}

\subsection{Simulation}
\label{sec:sim}
\input{sim}

\ifdefined\ASICSYNTH
\subsection{ASIC Synthesis}
\label{sec:synth}
\input{synth}
\fi

\ifdefined\FPGACOMP
\subsection{FPGA Compilation}
\label{sec:fpga}
\input{fpga}
\fi

\ifdefined\SECTIONCLEARPAGE
\clearpage
\fi
\subsection{End to End Examples}
\label{sec:examples}
\input{examples}

\subsection{Customizing Py2HWSW}
\label{sec:customizing_py2}
\input{customizing_py2}

\subsection{Troubleshooting}
\label{sec:troubleshooting}
\input{troubleshooting}

%
% Configuration
%
% \ifdefined\SECTIONCLEARPAGE
% \clearpage
% \fi
% \section{Configuration}
% \label{sec:config}
% \input{config}

% \subsection{Configuration Files}
% \label{sec:config_files}
% \input{config_files}

% \subsection{Command Line Options}
% \label{sec:command_line_options}
% \input{command_line_options}

% \subsection{Environment Variables}
% \label{sec:environment_vars}
% \input{environment_vars}

%
% Advanced Topics
%
% \ifdefined\SECTIONCLEARPAGE
% \clearpage
% \fi
% \section{Advanced Topics}
% \label{sec:advanced_topics}
% \input{advanced_topics}

% \subsection{Customizing the Framework}
% \label{sec:customizing_framework}
% \input{customizing_framework}

% \subsection{Creating Custom Classes}
% \label{sec:creating_custom_classes}
% \input{creating_custom_classes}


%
% Integrate Py2HWSW with external tools
%
% \section{Integrating with Other Tools}
% \label{sec:integrating_external_tools}
% \input{integrating_external_tools}

% \subsection{Integrate non-py2hwsw cores}
% \label{sec:non_py2_cores}
% \input{non_py2_cores}

% \subsection{Integrate py2hwsw cores in non-py2hwsw ecosystems}
% \label{sec:non_py2_ecosystems}
% \input{non_py2_ecosystems}

%
% Short Notation
%
% \section{Short Notation}
% \label{sec:short_notation}
% \input{short_notation}

%
% Appendices
%
% \ifdefined\SECTIONCLEARPAGE
% \clearpage
% \fi
% \section{Appendices}
% \label{sec:appendices}
% \input{appendices}

%\subsection{Glossary}
%\label{sec:glossary}
%\input{glossary}

%\subsection{Release Notes}
%\label{sec:release_notes}
%\input{release_notes}

%\subsection{Contributing}
%\label{sec:contributing}
%\input{contributing}

%\subsection{FAQ}
%\label{sec:faq}
%\input{faq}

%\subsection{Troubleshooting Guide}
%\label{sec:troubleshooting_guide}
%\input{troubleshooting_guide}

\end{document}
